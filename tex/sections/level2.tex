\section*{Επίπεδο 2}
Σε αυτό το επίπεδο λαμβάνουμε το σήμα εκφρασμένο ως συντελεστές \emph{MDCT} από
το πρώτο επίπεδο και εκτελούμε τη διαδικασία του \emph{Temporal Noise Shaping
(TNS)} ώστε να μειώσουμε το σφάλμα κβαντισμού σε μετέπειτα στάδιο. Να σημειωθεί
ότι η απώλεια πληροφόριας σε αυτό το στάδιο είναι πρακτικά μηδενική, συνεπώς
και σε αυτό το επίπεδο έχουμε lossless κωδικοποιήση. Το επίπεδο αυτό
αποτελείται από τα ακόλουθα αρχεία:


\subsection*{TNS.m}
Υλοποιεί τη βαθμίδα \emph{Temporal Noise Shaping (TNS)} για ένα κανάλι και έχει
το ακόλουθο signature:
\begin{center}
	\verb|[frameFout, TNScoeffs] = TNS(frameFin, frameType)|
\end{center}

\noindent Επιστρέφει τις ακόλουθες μεταβλητές:
\begin{description}
\item[\Q{frameFout:}] Οι συντελεστές \emph{MDCT} μετά τη διαδικασία του
	\emph{Temporal Noise Shaping}. Πίνακας διαστάσεων $128 \times 8$ αν το
	frame είναι \verb|ESH|, ειδάλως διάνυσμα $1024 \times 1$.
\item[\Q{TNScoeffs:}] Οι κβαντισμένοι συντελεστές \emph{TNS} σε μορφή πίνακα
	διαστάσεων $4 \times 8$ για \verb|ESH| frame, $4 \times 1$ αλλιώς.
\end{description}

\noindent Δέχεται τα ακόλουθα ορίσματα:
\begin{description}
\item[\Q{frameFin:}] Οι συντελεστές \emph{MDCT} πριν τη διαδικασία του
	\emph{Temporal Noise Shaping}. Πίνακας διαστάσεων $128 \times 8$ αν το
	frame είναι \verb|ESH|, ειδάλως διάνυσμα $1024 \times 1$.
\item[\Q{frameType:}] Ο τύπος του frame, όπως και παραπάνω.
\end{description}


\subsection*{iTNS.m}
Αντιστρέφει τη βαθμίδα \emph{Temporal Noise Shaping (TNS)} για ένα κανάλι και
έχει το ακόλουθο signature:
\begin{center}
	\verb|frameFout = iTNS(frameFin, frameType, TNScoeffs)|
\end{center}

\noindent Τα ορίσματα και οι μεταβλητές εξόδου έχουν όπως και προηγουμένως.


\subsection*{AACoder2.m}
Υλοποιεί όλη τη κωδικοποίηση του δεύτερου επιπέδου με το ακόλουθο signature:
\begin{center}
	\verb|AACSeq2 = AACoder2(fNameIn)|
\end{center}

\noindent Επιστρέφει τις ακόλουθες μεταβλητές:
\begin{description}
\item[\Q{AACSeq2:}] Δομή διαστάσεων $K \times 1$, όπου $K$ το πλήθος των
	κωδικοποιημένων frames. Κάθε στοιχείο της δομής αποτελείται από:
	\begin{description}[leftmargin=0.2cm,rightmargin=0cm]
		\item \Q{AACSeq2(i).frameType:} Όπως παραπάνω.
		\item \Q{AACSeq2(i).winType:} Όπως παραπάνω.
		\item \Q{AACSeq2(i).chl.frameF:} Οι συντελεστές \emph{(MDCT)} μετά τη
			διαδικασία \emph{TNS} για το αριστερό κανάλι διαστάσεων
			$128 \times 8$ για \verb|ESH| frames και $1024 \times 1$ για όλα τα
			άλλα.
		\item \Q{AACSeq2(i).chr.frameF:} Οι συντελεστές \emph{(MDCT)} μετά τη
			διαδικασία \emph{TNS} για το δεξί κανάλι διαστάσεων $128 \times 8$
			για	\verb|ESH| frames και $1024 \times 1$ για όλα τα άλλα.
		\item \Q{AACSeq2(i).chl.TNScoeffs:} Οι κβαντισμένοι συντελεστές
			\emph{TNS} για το αριστερό κανάλι. Πίνακας διαστάσεων $4 \times 8$
			για \verb|ESH| frame, $4 \times 1$ αλλιώς.
		\item \Q{AACSeq2(i).chr.TNScoeffs:} Οι κβαντισμένοι συντελεστές
			\emph{TNS} για το δεξί κανάλι. Πίνακας διαστάσεων $4 \times 8$
			για \verb|ESH| frame, $4 \times 1$ αλλιώς.
		\end{description}
\end{description}

\noindent Δέχεται τα ακόλουθα ορίσματα:
\begin{description}
\item[\Q{fNameIn:}] Το όνομα του αρχείου \verb|wav| με το σήμα προς
	κωδικοποίηση. Υποθέτουμε δικάναλο ήχο και δειγματοληψία 48 kHz.
\end{description}


\subsection*{iAACoder2.m}
Υλοποιεί όλη την αποκωδικοποίηση του δεύτερου επιπέδου με το ακόλουθο
signature:
\begin{center}
	\verb|x = iAACoder2(AACSeq2, fNameOut)|
\end{center}

\noindent Επιστρέφει τις ακόλουθες μεταβλητές:
\begin{description}
  \item[\Q{x:}] Οι αποκωδικοποιημένες τιμές του σήματος.
\end{description}

\noindent Δέχεται τα ακόλουθα ορίσματα:
\begin{description}
\item[\Q{AACSeq2:}] Όπως και παραπάνω.
\item[\Q{fNameOut:}] Το όνομα του αρχείου \verb|wav| στο οποίο θα εγγραφεί το
	σήμα ήχου μετά την αποκωδικοποίηση. Υποθέτουμε δικάναλο ήχο και
	δειγματοληψία 48 kHz.
\end{description}

\subsection*{demoAAC2.m}
Επιδεικνύει τη διαδικασία κωδικοποίησης/αποκωδικοποίησης του δεύτερου επιπέδου
εκτυπώνοντας χρήσιμες πληροφορίες με το ακόλουθο signature:
\begin{center}
	\verb|SNR = demoAAC2(fNameIn, fNameOut)|
\end{center}

\noindent Επιστρέφει τις ακόλουθες μεταβλητές:
\begin{description}
\item[\Q{SNR:}] Signal-to-Noise ratio μεταξύ του αποκωδικοποιημένου σήματος και
	της διαφοράς του με το αρχικό σε dB.
\end{description}

\noindent Δέχεται τα ακόλουθα ορίσματα:
\begin{description}
\item[\Q{fNameIn:}] Όπως και παραπάνω.
\item[\Q{fNameOut:}] Όπως και παραπάνω.
\end{description}
