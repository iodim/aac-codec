\section*{Επίπεδο 1}
Το επίπεδο αυτό αρχικά χωρίζει το ηχητικό σήμα σε frames μήκους 2048 στοιχείων
έκαστο, με 50\% overlap. Για να καλύψουμε τυχόν προβλήματα σε οριακές συνθήκες
(στην αρχή και το τέλος του σήματος) κάνουμε pad με frames με μηδενικά, μισό (1024)
στην αρχή και ένα (2048) στο τέλος. Επίσης, αν το σήμα δεν αποτελείται από ακέραιο
πολλαπλάσιο του 1024, προσθέτουμε μηδενικά μέχρι να πετύχουμε τέλεια διαίρεση.
Το επίπεδο αυτό υλοποιεί τη βαθμίδα του \emph{Sequence Segmentation Control
(SSC)}, η οποία ουσιαστικά κατηγοριοποιεί το κάθε frame ανάλογα με το
ενεργειακό του περιεχόμενο σε \verb|ONLY_LONG_SEQUENCE| (\verb|OLS|),
\verb|LONG_START_SEQUENCE| (\verb|OLS|), \verb|EIGHT_SHORT_SEQUENCE|
(\verb|OLS|) και \verb|LONG_STOP_SEQUENCE (LPS)| (\verb|OLS|). Επιπλέον,
υλοποιείται και η βαθμίδα των \emph{Filterbanks} στην οποία περνάμε το κάθε
frame ή subframe από ένα παράθυρο, είτε \emph{Kaiser Bessel Derived (KBD)} είτε
\emph{Sinusoid (SIN)}, και έπειτα εξάγουμε τους συντελεστές του μετασχηματισμού
\emph{Modified Discrete Cosine Transform (MDCT)} των δειγμάτων. Επιπλέον
υλοποιούνται και οι αντίστροφες συναρτήσεις για κάθε στάδιο που θα
χρησιμοποιηθούν κατά τη διαδικασία της αποκωδικοποιήσης. Να σημειωθεί ότι η
κωδικοποιήση αυτού του επίπεδου είναι lossless καθώς δεν αποκόπτεται καμία
πληροφορία, απλά μεταβαίνουμε από το χρόνο στη συχνότητα και το αντίστροφο.
Δε θα εμβαθύνουμε ιδιαίτερα στη λειτουργία των μεθόδων αυτών, καθώς
επεξηγούνται επαρκώς στην εκφώνηση. Το επίπεδο αυτό αποτελείται από τα
ακόλουθα αρχεία:


\subsection*{SSC.m}
Υλοποιεί τη βαθμίδα \emph{Sequence Segmentation Control (SSC)} και έχει το
ακόλουθο signature:
\begin{center}
	\verb|frameType = SSC(frameT, nextFrameT, prevFrameType)|
\end{center}

\noindent Επιστρέφει τις ακόλουθες μεταβλητές:
\begin{description}
\item[\Q{frameType:}] Ο τύπος του frame που επιλέγεται. Για \verb|OLS|
	χρησιμοποιούμε την τιμή 1, για 	\verb|LSS| την τιμή 2, για \verb|ESH| την
	τιμή 3 και για \verb|LPS| την τιμή 4.
\end{description}

\noindent Δέχεται τα ακόλουθα ορίσματα:
\begin{description}
\item[\Q{frameT:}] Το frame $i$ στο πεδίο του χρόνου. Περιέχει 2 κανάλια.
	Πίνακας διαστάσεων $2048 \times 2$.
\item[\Q{nextFrameT:}] Το επόμενο frame, $i + 1$. Χρησιμοποιείται για την
	επιλογή παραθύρου. Πίνακας διαστάσεων $2048 \times 2$.
\item[\Q{prevFrameType:}] Ο τύπος που έχει επιλεγεί για το frame $i - 1$. Να
	σημειωθεί ότι μπορεί να πάρει και την τιμή 0, η οποία χρησιμοποιείται όταν
	πρέπει να αποφασίσουμε την τιμή του πρώτου frame.
\end{description}


\subsection*{filterbank.m}
Υλοποιεί τη βαθμίδα \emph{Filterbank} και έχει το ακόλουθο function signature:
\begin{center}
	\verb|frameF = filterbank(frameT, frameType, winType)|
\end{center}

\noindent Επιστρέφει τις ακόλουθες μεταβλητές:
\begin{description}
\item[\Q{frameF:}] Το frame στο πεδίο της συχνότητας, αναπαριστόμενο από τους
	συντελεστές \emph{MDCT}. Πίνακας διαστάσεων $1024 \times 2$  που περιέχει
	είτε τους συντελεστές των δύο καναλίων όταν έχουμε \verb|OLS|, \verb|LSS|
	ή \verb|LPS|, είτε 8 υποπίνακες διαστάσεων $128 \times 2$, έναν για κάθε
	subframe, όταν έχουμε \verb|ESH|, τοποθετημένους σε στήλες.
\end{description}

\noindent Δέχεται τα ακόλουθα ορίσματα:
\begin{description}
\item[\Q{frameT:}] Το frame $i$ στο πεδίο του χρόνου.Περιέχει 2 κανάλια.
	Πίνακας διαστάσεων $2048 \times 2$.
\item[\Q{frameType:}] Ο τύπος που έχει επιλεχθεί για το υπο κωδικοποίηση frame.
\item[\Q{winType:}] Ο τύπος παραθύρου για το τρέχον frame. Μπορεί να είναι είτε
	\emph{Kaiser Bessel Derived (KBD)} που συμβολίζεται με την τιμή 5, είτε
	\emph{Sinusoid (SIN)} που συμβολίζεται με την τιμή 6.
\end{description}


\subsection*{iFilterbank.m}
Υλοποιεί την βαθμίδα που αντιστρέφει την βαθμίδα \emph{Filterbank} και έχει το
ακόλουθο function signature:
\begin{center}
	\verb|frameT = iFilterbank(frameF, frameType, winType)|
\end{center}

\noindent Τα ορίσματα και οι μεταβλητές εξόδου έχουν όπως και προηγουμένως.


\subsection*{mdct4.m}
Υπολογίζει το μετασχηματισμό \emph{Modified Discrete Cosine Transform (MDCT)}. Η
υλοποίηση έχει γίνει από το Μάριο Αθηναίο
\footnote{\label{f1}\url{http://www.ee.columbia.edu/~marios/mdct/mdct_giraffe.html}}.


\subsection*{imdct4.m}
Υπολογίζει τον αντίστροφο μετασχηματισμό \emph{Modified Discrete Cosine
Transform (MDCT)}. Η υλοποίηση έχει γίνει από το Μάριο Αθηναίο\footref{f1}.


\subsection*{kbdwin.m}
Υπολογίζει τους συντελεστές ενός παραθύρου \emph{Kaiser Bessel Derived (KBD)}
με το ακόλουθο signature:
\begin{center}
	\verb|[WL, WR] = kbdwin(N, alpha)|
\end{center}

\noindent Επιστρέφει τις ακόλουθες μεταβλητές:
\begin{description}
\item[\Q{WL:}] Οι συντελεστές του αριστερού κομματιού του παραθύρου, μήκους
	$\frac{N}{2}$.
\item[\Q{WR:}] Οι συντελεστές του δεξιού κομματιού του παραθύρου, μήκους
	$\frac{N}{2}$.
\end{description}

\noindent Δέχεται τα ακόλουθα ορίσματα:
\begin{description}
\item[\Q{N:}] Το επιθυμητό μήκος του παραθύρου.
\item[\Q{alpha:}] Παράμετρος που επηρεάζει τη μορφή του παραθύρου.
\end{description}


\subsection*{sinwin.m}
Υπολογίζει τους συντελεστές ενός παραθύρου \emph{Sinusoid (SIN)} με το ακόλουθο
signature:
\begin{center}
	\verb|[WL, WR] = sinwin(N)|
\end{center}

\noindent Τα ορίσματα και οι μεταβλητές εξόδου έχουν όπως και προηγουμένως.


\subsection*{AACoder1.m}
Υλοποιεί όλη τη κωδικοποιήση του πρώτου επιπέδου με το ακόλουθο signature:
\begin{center}
	\verb|AACSeq1 = AACoder1(fNameIn)|
\end{center}

\noindent Επιστρέφει τις ακόλουθες μεταβλητές:
\begin{description}
\item[\Q{AACSeq1:}] Δομή δίαστάσεων $K \times 1$, όπου $K$ το πλήθος των
	κωδικοποιημένων frames. Κάθε στοιχείο της δομής αποτελείται από:
	\begin{description}[leftmargin=0.2cm,rightmargin=0cm]
		\item \Q{AACSeq1(i).frameType:} Όπως παραπάνω.
		\item \Q{AACSeq1(i).winType:} Όπως παραπάνω.
		\item \Q{AACSeq1(i).chl.frameF:} Οι συντελεστές \emph{(MDCT)} για το
		αριστερό κανάλι διαστάσεων $128 \times 8$ για \verb|ESH| frames και
		$1024 \times 1$ για	όλα τα άλλα.
		\item \Q{AACSeq1(i).chr.frameF:} Οι συντελεστές \emph{(MDCT)} για το
		δεξί κανάλι διαστάσεων $128 \times 8$ για \verb|ESH| frames και
		$1024 \times 1$ για όλα τα άλλα.
	\end{itemize}
\end{description}

\noindent Δέχεται τα ακόλουθα ορίσματα:
\begin{description}
\item[\Q{fNameIn:}] Το όνομα του αρχείου \verb|wav| με το σήμα προς
	κωδικοποιήση. Υποθέτουμε δικάναλο ήχο και δειγματοληψία 48 kHz.
\end{description}


\subsection*{iAACoder1.m}
Υλοποιεί όλη την απόκωδικοποιήση του πρώτου επιπέδου με το ακόλουθο signature:
\begin{center}
	\verb|x = iAACoder1(AACSeq1, fNameOut)|
\end{center}

\noindent Επιστρέφει τις ακόλουθες μεταβλητές:
\begin{description}
\item[\Q{AACSeq1:}] Όπως και παραπάνω.
\item[\Q{fNameOut:}] Το όνομα του αρχείου \verb|wav| στο οποίο θα εγγραφεί το
	σήμα ήχου μετά την αποκωδικοποιήση. Υποθέτουμε δικάναλο ήχο και
	δειγματοληψία 48 kHz.
\end{description}

\noindent Δέχεται τα ακόλουθα ορίσματα:
\begin{description}
\item[\Q{fNameIn:}] Το όνομα του αρχείου \verb|wav| με το σήμα προς
	κωδικοποιήση. Υποθέτουμε δικάναλο ήχο και δειγματοληψία 48 kHz.
\end{description}


\subsection*{demoAAC1.m}
Επιδεικνύει τη διαδικασία κωδικοποιήσης/αποκωδικοποίησης του πρώτου επιπέδου
εκτυπώνοντας χρήσιμες πληροφορίες με το ακόλουθο signature:
\begin{center}
	\verb|SNR = demoAAC1(fNameIn, fNameOut)|
\end{center}

\noindent Επιστρέφει τις ακόλουθες μεταβλητές:
\begin{description}
\item[\Q{SNR:}] Signal-to-Noise ratio μεταξύ του αποκωδικοποιημένου σήματος και
	της διαφοράς του με το αρχικό σε dB.
\end{description}

\noindent Δέχεται τα ακόλουθα ορίσματα:
\begin{description}
\item[\Q{fNameIn:}] Όπως και παραπάνω.
\item[\Q{fNameOut:}] Όπως και παραπάνω.
\end{description}
