\section*{Πειραματικά αποτελέσματα και συμπεράσματα}
Σε αυτήν την ενότητα θα παρουσιάσουμε τα αποτελέσματα που επιτυγχάνουμε σε
χρόνους διαδικασιών, \verb|Signal-to-Noise-Ratio|, bitrate και συμπίεση για τα
τρία επίπεδα του κωδικοποιητή που δημιουργήσαμε σε αυτήν την εργασία. Για την
εξαγωγή των αποτελεσμάτων χρησιμοποιήθηκε το Matlab script \emph{demo.m}, το
οποίο δημιουργήθηκε για αυτόν τον σκοπό. Συνοπτικά, τα αριθμητικά αποτελέσματα
φαίνονται στον παρακάτω πίνακα.

\usepackage{multirow}

\begin{center}
  \begin{tabular}{l | l | l | l}
    \multicolumn{4}{c}{\emph{Results}} \\ \hline
      & Level 1 & Level 2 & Level 3 \\ \hline
    Encoding Time & 1.1239 s & 1.7551 s & 151.7964 s \\ \hline
    Decoding Time & 0.3670 s & 0.2318 s & 5.1134 s \\ \hline
    Channel 1 SNR & 307.87 dB & 307.86 dB & 6.52 dB \\ \hline
    Channel 2 SNR & 307.93 dB & 307.91 dB & 6.38 dB \\ \hline
    Uncompressed audio size & \multicolumn{3}{c}{1.0795 MB} \\ \hline
    Compressed struct size & 4330.33 KB & 4358.23 KB & 224.13 KB \\ \hline
    Uncompressed audio bitrate & \multicolumn{3}{c}{187.51 KB/s} \\ \hline
    Compressed struct bitrate & 734.53 KB/s & 739.26 KB/s & 38.02 KB/s \\ \hline
    Compression rate & $\times 0.255$ & $\times 0.253$ & $\times 4.93$ \\
  \end{tabular}
\end{center}

Όπως παρατηρούμε από τον πίνακα αποτελεσμάτων, η κωδικοποίηση του αρχικού σήματος
ήχου με τα δύο πρώτα επίπεδα δίνει πολύ υψηλό \verb|SNR|, το οποίο είναι αναμενόμενο
καθώς η Level 1 διαδικασία είναι μη-απωλεστική (lossless) και η Level 2 πρακτικά
μη-απωλεστική. Οι δομές που προκύπτουν από αυτές τις διαδικασίες είναι μεγαλύτερες
σε μέγεθος της αρχικής, καθώς αποθηκεύουν κάποια μετασχηματισμένη ακολουθία του
αρχικού σήματος (συντελεστές \emph{MDCT}) και επιπλέον πληροφορίες που χρειάζονται
για την αντιστροφή. Δεν εφαρμόζεται κάποιος αλγόριθμος συμπίεσης σε αυτά τα δεδομένα,
π.χ. \emph{Huffman}.

Η συμπίεση του αρχικού ηχητικού σήματος επιτυγχάνεται ουσιαστικά στην δομή που
προκύπτει από το επίπεδο 3 της διαδικασίας κωδικοποίησης. Οφείλεται στην εφαρμογή
των \emph{κωδικό-λεξικών Huffman} του προτύπου \emph{AAC} στα έξυπνα κβαντισμένα
δεδομένα εξαιτίας της χρήσης του \emph{ψυχοακουστικού μοντέλου}. Η ακουστική
διαφορά που έχουν ο αρχικός ήχος με τον αποκωδικοποιήμενο ήχο από το αρχείο επιπέδου
3 είναι ανεπαίσθητη, όπως ακριβώς περιμέναμε. Ενώ όμως πετυχαίνουμε συμπίεση περίπου
5:1, ο χρόνος κωδικοποίησης και αποκωδικοποίησης έχουν αυξηθεί. Αυτό δεν προκαλεί
επίσης έκπληξη καθώς αποτελεί παράδειγμα του αντάλλαγματος χώρου αποθήκευσης -
χρόνου κωδικοποίησης, το οποίο διέπει τις διαδικασίες συμπίεσης. Παρ'όλ'αυτά η
αύξηση δεν είναι τέτοια ώστε να αποτελεί πρόβλημα, ειδικά στην αποκωδικοποίηση
όπου ο χρόνος είναι πιο σημαντικός.
